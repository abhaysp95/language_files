% function are called with class
\documentclass{article}	% some others are book, a4letter, beamer
\usepackage{graphicx}	% for images
\usepackage{blindtext}
\usepackage{wrapfig}

\author{Abhay Shanker Pathak}
\title{First {\LaTeX} document}
\date{Last Tuesday}

% stuff up this line is preamble
\begin{document}

\maketitle

\section{Introduction}

	\blindtext
	\blindtext
	\blindtext
% it's going in next section
	\begin{figure}[h]
	\centering
		\includegraphics[width=0.7\textwidth]{/home/raytracer/Pictures/Screenshots/2020-01-28-195626_1366x768_scrot.png}
		\caption{tiling window manager}
	\end{figure}
	\blindtext
	\blindtext

	This is some text.
	This new line in my editor.

	You have to make two returns.

\section{new section}

So, in Section~\ref{list}, we will talk about lists

\begin{center}
\includegraphics[width=5in,height=6in,keepaspectratio]{/home/raytracer/Pictures/Screenshots/2020-02-12-203212_1338x722_scrot.png}
\end{center}

\section{Just another section}

\begin{figure}[h]	% h to put exactly where you want
	% t to put image on top of page, b to put on bottom and p to put on page of its own
\begin{center}
\includegraphics[scale=0.3]{$HOME/Pictures/Screenshots/2020-02-11-171559_1366x768_scrot.png}
\end{center}
	\caption{example: .xinitrc}
\end{figure}

We talked about bread, in Section\textbf{~\ref{list}} on number\emph{~\ref{bread}}

\section{Formatting}

\blindtext
\blindtext
\begin{wrapfigure}{l}{4in}
%\begin{center}
	\centering
	\includegraphics[width=3in]{$HOME/Pictures/Screenshots/2020-02-21-111530_1366x768_scrot.png}
	\caption{compton effect\label{compton}}
%\end{center}
\end{wrapfigure}
\blindtext
\blindtext
\blindtext
\blindtext
\blindtext

	\subsection{Subsection}

	I'm going to put a image here

\begin{center}
\includegraphics[width=0.8\textwidth,angle=5]{$HOME/Pictures/Screenshots/2020-02-12-203212_1338x722_scrot.png}
\end{center}

		This is normal text. \textbf{This is bold text}. \textit{This is italic!}. \emph{This is emphatic.}

		Refer, to \textbf{Figure: }~\ref{compton}, to get to know \textbf{COMPTON} effect

		\underline{Here are some underlined text.}

		% "This is in quotation marks."
		% makes opening and closing both closing. So, use grave

		`` This is in proper quotation marks.''

		` This is in single quotation marks.'

\section{Lists\label{list}}

This is an ordered list

\begin{enumerate}
	\item Bread\label{bread}
	\item butter
	\item toilet paper
	\item cream
\end{enumerate}

This is an unordered list

\begin{itemize}
		\item corn
			\begin{itemize}
				\item red corn
				\item black corn
			\end{itemize}
		\item maize
		\item lamp
		\item bulb
\end{itemize}
\end{document}
